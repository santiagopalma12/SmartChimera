\documentclass[conference]{IEEEtran}
\IEEEoverridecommandlockouts
% Packages
\usepackage{cite}
\usepackage{amsmath,amssymb,amsfonts}
\usepackage{algorithmic}
\usepackage{graphicx}
\usepackage{textcomp}
\usepackage{xcolor}
\usepackage{url}
\usepackage{booktabs}
\usepackage{subcaption}

\def\BibTeX{{\rm B\kern-.05em{\sc i\kern-.025em b}\kern-.025em
    T\kern-.1667em\lower.7ex\hbox{E}\kern-.125emX}}

\begin{document}

\title{SmartChimera: Sistema de Ensamblaje de Equipos de Desarrollo de Software usando TAD Grafo y Análisis de Riesgo Híbrido}

\author{\IEEEauthorblockN{Santiago Palma}
\IEEEauthorblockA{\textit{Escuela Profesional de Ingeniería de Sistemas} \\
\textit{Universidad Nacional de San Agustín}\\
Arequipa, Perú \\
spalma@unsa.edu.pe}
}

\maketitle

\begin{abstract}
El Bus Factor representa el número mínimo de desarrolladores cuya salida simultánea colapsaría un proyecto de software, constituyendo un riesgo crítico para la continuidad organizacional. Este paper presenta SmartChimera, un sistema que utiliza Teoría de Grafos para mitigar este riesgo sin depender de "Cajas Negras" de Inteligencia Artificial. A diferencia de enfoques que requieren Big Data histórico, SmartChimera implementa una arquitectura determinística basada en dos motores: (1) Un detector de "Linchpins" híbrido que combina métricas de centralidad (Brandes) con pesos de dependencia de proyecto, y (2) Un algoritmo de ensamblaje de equipos guiado por perfiles de misión (Resilient vs Growth). Validamos el sistema mediante una simulación de escenarios corporativos con arquetipos de riesgo diseñados, demostrando cómo la topología del grafo permite identificar vulnerabilidades estructurales que las métricas de desempeño tradicionales ignoran.
\end{abstract}

\begin{IEEEkeywords}
Bus Factor, Team Formation, Graph Theory, Risk Management, Software Engineering, Heuristics, Simulation.
\end{IEEEkeywords}

\section{Introducción}

El desarrollo de software moderno enfrenta un problema crítico: la dependencia excesiva en desarrolladores clave. El \textit{Bus Factor} (BF), introducido por Avelino et al. \cite{avelino2016novel}, mide el riesgo de concentración de conocimiento. Estudios empíricos revelan que el 50\% de proyectos open-source tienen BF$\leq$2 \cite{avelino2016novel}, exponiendo una vulnerabilidad sistémica.

La rotación de desarrolladores (turnover) amplifica este riesgo. Foucault et al. \cite{foucault2015impact} demuestran que la salida de core developers incrementa defectos en 40-60\%. Lin et al. \cite{lin2017developer} reportan que la pérdida de conocimiento tácito \cite{ryan2013acquiring} genera disrupciones severas en la arquitectura del sistema \cite{maccormack2012exploring}. La coordinación de expertos en redes sociales \cite{meneely2008predicting} y la propiedad colectiva del código \cite{maruping2009role} son factores mitigantes que a menudo se ignoran.

La literatura actual aborda este problema mediante optimización NP-hard \cite{lappas2009finding} o modelos predictivos de Machine Learning que requieren datasets masivos e inmaculados, poco realistas para la mayoría de empresas \cite{saeedi2025survey, ma2020data}. SmartChimera llena este vacío proponiendo una solución basada en \textbf{Heurísticas Estructurales} y grafos de colaboración explícitos \cite{caglayan2013emergence}, eliminando la necesidad de entrenamiento de modelos y ofreciendo auditabilidad total.

\section{Objetivos}

\subsection{Objetivo General}
Diseñar e implementar un sistema de ensamblaje de equipos que minimice el Bus Factor mediante el uso de algoritmos de grafos y perfiles de misión configurables, validado en entornos simulados de alta fidelidad.

\subsection{Objetivos Específicos}
\begin{enumerate}
    \item \textbf{OE1:} Implementar un modelo de red de colaboración $G(V,E)$ ponderado que integre interacciones sociales y carga de proyectos, siguiendo a \cite{joblin2017evolutionary}.
    \item \textbf{OE2:} Desarrollar un algoritmo de detección de riesgo híbrido que pondere \textit{Betweenness Centrality} (Brandes \cite{brandes2001faster}) y \textit{Project Weight} para identificar Linchpins ocultos, superando las limitaciones de métricas puramente topológicas \cite{saxena2020centrality}.
    \item \textbf{OE3:} Construir un motor de formación de equipos "Dual-Mode" capaz de alternar entre estrategias de Resilencia (Minimizar Riesgo) e Innovación (Maximizar Expertise) según el ciclo de vida del software \cite{ulziit2015conceptual}.
\end{enumerate}

\section{Justificación}

\subsection{Relevancia Industrial}
Robillard \cite{robillard2021turnover} reporta que el 80\% de managers en Microsoft consideran el turnover como el mayor riesgo técnico. Casos como el incidente de \textit{left-pad} (2016) o el bug \textit{Heartbleed} (OpenSSL, BF$\approx$2) demuestran que la falta de redundancia es catastrófica. La adopción de DevOps \cite{lwakatare2019devops} requiere una gestión proactiva del conocimiento.

\subsection{Gap Técnico}
Mientras la industria adopta herramientas de gestión ágil (JIRA), carece de instrumentos para medir la "salud estructural" del equipo. Las soluciones académicas basadas en AI son opacas ("Black Box") \cite{fathian2017new}. Nuestra propuesta determinística ofrece transparencia: un gerente puede ver por qué un equipo es riesgoso ("Todos dependen de nodo $A$").

\section{Metodología}

\subsection{Modelo de Grafo y Riesgo Híbrido}

Modelamos la organización como un grafo $G = (V, E)$, donde $V$ son empleados y $E$ colaboraciones \cite{mcclean2021social}. Para superar las limitaciones de la centralidad pura, introducimos una métrica de riesgo compuesta:

\begin{equation}
\label{eq:risk}
Risk(v) = \alpha \cdot \frac{BC(v) - \min(BC)}{\max(BC) - \min(BC)} + \beta \cdot \frac{Projects(v)}{\max(Projects)}
\end{equation}

Donde $BC(v)$ se calcula optimizando el algoritmo de Brandes \cite{brandes2001faster}:
\begin{equation}
BC(v) = \sum_{s \neq v \neq t \in V} \frac{\sigma_{st}(v)}{\sigma_{st}}
\end{equation}

Esta formulación híbrida ($\alpha=\beta=0.5$) permite detectar:
\begin{enumerate}
    \item \textbf{Social Hubs:} Conectores vitales para la comunicación (Alto $BC$).
    \item \textbf{Workhorses:} Expertos aislados con alta carga de mantenimiento (Alto $PW$).
\end{enumerate}

\subsection{Motor de Formación "Dual-Mode"}

El sistema adapta la topología del equipo al \textit{Perfil de Misión} \cite{li2015replacing}:

\subsubsection{Modo Resiliente (Safe Bet)}
Minimiza el riesgo operativo maximizando la redundancia funcional:
\begin{equation}
\text{Maximize } \sum_{skill \in S} (|Team(skill)| - 1)
\end{equation}
Sujeto a $BC(Team) < Threshold_{critical}$.

\subsubsection{Modo Crecimiento (Innovation)}
Prioriza la profundidad técnica ($SkillDepth$), permitiendo la inclusión de "Linchpins" para resolver problemas complejos, aceptando el riesgo asociado \cite{dave2018combined}.

\section{Desarrollo}

\subsection{Arquitectura del Sistema}

La implementación sigue una arquitectura de microservicios sobre Docker, diseñada para escalabilidad y modularidad \cite{monteiro2023experimental}:

\begin{figure}[h]
\centering
\fbox{\includegraphics[width=0.45\textwidth]{figuras/fig1_arquitectura.png}} % Placeholder
\caption{Arquitectura de SmartChimera: (1) Graph Data Layer (Neo4j), (2) Analysis Engine (NetworkX), (3) API Gateway (FastAPI), (4) Reactive UI (React)}
\label{fig:arquitectura}
\end{figure}

\begin{itemize}
    \item \textbf{Capa de Persistencia (Graph Layer):} Utilizamos Neo4j 5.x \cite{fernandes2018graph} para almacenar relacionalidades nativas. Los nodos `Empleado` se conectan mediante aristas `TRABAJO_CON` ponderadas por la frecuencia de interacción en repositorios (Git commits).
    \item \textbf{Motor de Análisis (Core):} Escrito en Python, implementa el algoritmo de Brandes optimizado con paralelización para grafos densos ($O(V \cdot E)$).
    \item \textbf{API & UI:} FastAPI expone endpoints RESTful para la consulta de métricas en tiempo real, consumidos por un frontend React que visualiza el grafo usando `d3-force`.
\end{itemize}

\section{Validación y Caso de Estudio}

Diseñamos una simulación determinística con 150 nodos y 12 arquetipos de riesgo, basada en patrones industriales \cite{lin2017developer}.

\subsection{Escenarios Evaluados}

\subsubsection{Escenario A: "El Héroe Cansado" (High Project Weight)}
Nodo "Carlos Ruiz" (45 proyectos, baja centralidad).
\begin{itemize}
    \item \textbf{Resultado:} Clasificado como \textbf{CRITICAL} por la componente $\beta \cdot PW(v)$ de la Ec. \ref{eq:risk}. Brandes puro ($BC$) falló en detectarlo.
\end{itemize}

\subsubsection{Escenario B: "El Puente Humano" (High Centrality)}
Nodo "Ana Silva" (Pocos proyectos, alta intermediación).
\begin{itemize}
    \item \textbf{Resultado:} Clasificada como \textbf{HIGH RISK} ($BC > 0.6$). Acciones recomendadas: Pair programming obligatorio.
\end{itemize}

\subsection{Resultados Agregados}
Comparando equipos generados por SmartChimera (Modo Resiliente) vs. Asignación Aleatoria:
\begin{table}[h]
\centering
\caption{Comparación de Resiliencia (N=50 equipos)}
\begin{tabular}{@{}lcc@{}}
\toprule
\textbf{Métrica} & \textbf{Aleatorio} & \textbf{SmartChimera} \\ \midrule
Bus Factor Promedio & 1.2 $\pm$ 0.4 & \textbf{2.8 $\pm$ 0.5} \\
Skill Coverage & 85\% & \textbf{98\%} \\
Riesgo Máximo (Score) & 0.85 & \textbf{0.45} \\ \bottomrule
\end{tabular}
\end{table}

\section{Conclusiones}

SmartChimera demuestra que la mitigación del Bus Factor es posible mediante heurísticas estructurales transparentes, sin necesidad de cajas negras de IA. Integrando métricas de centralidad \cite{brandes2001faster} con carga de proyecto, ofrecemos una herramienta auditable para la gestión del riesgo de conocimiento en ingeniería de software.

\bibliographystyle{IEEEtran}
\bibliography{referencias_50}

\end{document}
